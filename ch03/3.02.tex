\problem{试求$SSSS$的$\bnf$.}
\begin{solution}
{\color {red} {对第一版答案做出补充说明.}}根据定义, $S \equiv \lambda xyz.xz(yz)$.注意:对于$SS$,第一个$S$和第二个$S$的变元不同.不妨令第一个$S$为$S \equiv \lambda xyz.xz(yz)$,第一个$S$为$S \equiv \lambda abf.af(bf)$.首先化简$SS$,再化简$SSSS$.

那么
    \begin{align*}
        SS&\equivbeta \lambda yz.Sz(yz)\\
        &\equivbeta \lambda yz.\ (\lambda abf.af(bf)) \ z(yz)\\
        &\equivbeta \lambda yz.\lambda f.zf((yz)f)\\
        &\equiv \lambda yzf.zf(yzf)\\
        &\equiv \lambda xyz.yz(xyz)
    \end{align*}

接下来再求$SSSS$的值,由于$SSSS\equivbeta SS(SS)$,一直进行规约化简有(注意第一个$SS$和第二个$SS$的变元不同):
    \begin{align*}
        SSSS &\equivbeta SS(SS)\\
        &\equivbeta \lambda yz.yz(SSyz)\\
        &\equivbeta \lambda yz.yz(\lambda f.zf(yzf))\\
        &\equiv \lambda xy.xy(\lambda z.yz(xyz))\tag{*}
    \end{align*}

    到了(*)式后, 已经没有含有$\beta$-可约式的子项, 故(*)式即为$SSSS$的$\bnf$.
\end{solution}