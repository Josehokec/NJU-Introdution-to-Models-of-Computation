\documentclass[a4paper,twoside]{article}
\usepackage[margin=1.25in]{geometry}
\usepackage[UTF8,fontset=none]{ctex}
\usepackage{amssymb}
\usepackage{amsthm}
\usepackage{titlesec}
\usepackage{fancyhdr}
\usepackage{listings}
\usepackage{indentfirst}
\usepackage{caption}
\usepackage{tabularx}
\usepackage{float}
\usepackage{longtable}
\usepackage {color}
%\setlength{\parindent}{2em}
\input{config.tex}
\setCJKmainfont[Path=fonts/, BoldFont=simhei.ttf, ItalicFont=simkai.ttf]{simsun.ttf}
\makeatletter
\def\thickhline{%
  \noalign{\ifnum0=`}\fi\hrule \@height \thickarrayrulewidth \futurelet
   \reserved@a\@thickhline}
\def\@thickhline{\ifx\reserved@a\thickhline
               \vskip\doublerulesep
               \vskip-\thickarrayrulewidth
             \fi
      \ifnum0=`{\fi}}
\makeatother

\newlength{\thickarrayrulewidth}
\setlength{\thickarrayrulewidth}{2\arrayrulewidth}
\newcolumntype{Y}{>{\centering\arraybackslash}X}
% Page head
\pagestyle{fancy}
\renewcommand{\sectionmark}[1]{\markright{#1}}
\fancyhead{} % clear all fields
\fancyhead[CO]{《计算模型导引》教材习题解答}
\fancyhead[CE]{第{\thesection}章 \leftmark}
\fancyhead[LE,RO]{\thepage}
\fancyfoot{}

% Title
\title{\bf 《计算模型导引》教材习题解答}
\author{Haitao Huang, Shizhe Liu}
\date{}
\begin{document}

\begin{titlepage}
\maketitle
\thispagestyle{empty}
\end{titlepage}
\begin{center}
\bf\LARGE 第一版声明
\end{center}

本习题解答与南京大学宋方敏教授编著的《计算模型导引》配套。

所有习题解答都是经过本人的思考,与周围人的讨论,以及来自大佬的讲解得到,所以整个题解并未完成所有的题目,也不能保证答案的严谨性和正确性,因此仅供参考。由于使用此答案造成的任何后果,请自行承担,本人概不负责。

本解答中许多灵感启发来自于蒋炎岩前辈提供的一份英文版参考解答,张强同学、李浩同学的解答,以及来自助教的解答,在此表示特别感谢。

同时殷切盼望各位大佬能对这份解答做出哪怕一点微小的贡献,包括但不限于:尚未解出题目的解答,不同于已有解答的另外的解题方法,对已有解答描述不够严谨甚至是错误的部分的补充、纠正或指出,对已有解答推理或描述不够清晰部分的补充或指出,公式中的错误,错别字等。任何的补充指正和提问,都感激不尽。

\rightline{作者:Haitao Huang}    %右对齐
\rightline{2020年于南京大学}


\begin{center}
\bf\LARGE 第一版补充说明
\end{center}

感谢Haitao Huang花费了巨大心血做出来教材习题解答第一版。第二版习题解答对第一版习题解答的第三章和五章解答做出了小的补充和修定。因此这个补充修订版的贡献量与第一版作者贡献量相比如同萤火虫与皓月的差距,因此在这向Haitao Huang大佬致敬!

感谢宋方敏老师传授向无知的我传授计算模型知识!在没学这门课程之前,我不懂什么是“计算”,“计算的范围有多大”,也不敢说我是学过计算机的。尽管我这门课我只懂一些皮毛,很多知识我还没有完全明白、吃透,但是我仍然觉得这门课很有价值,有意义!希望南大以后能够一直将这门必修课开设下去,给学生们涨涨“内力”,让学生们体会到“计算”之美。

感谢Liangchuan Luo一直无私地为我解答相关问题,感谢Jingsong Dai和Feng Chen帮我核验个人的解答过程。

由于个人并不精通本门课程,没有掌握其精髓,加之时间有限,习题解答中一定有疏漏和不足之处,希望读者批评指正。

\rightline{作者:Shizhe Liu}    %右对齐
\rightline{2022年于南京大学}


\thispagestyle{empty}
\newpage
\thispagestyle{empty}
\setcounter{page}{1}
% Chapter 1
\section{递归函数}
\problem{证明: 对于固定的$k$, 一元数论函数$x+k\in\BF$.}
\begin{proof}{\color {red} {对第一版答案做出补充说明.}}

\textbf{证法一}:因为$S(x)=x+1$(page 1 定义1.2),记$S^k \circ (x) = \underbrace{S\circ S \circ \cdots \circ S(x)}_{k\text{个}S}$,其中$\circ$是定义1.4的复合算子(page 4),规定$k=0$时,$S^k \circ (x)=x$.

由定理1.2(page 6)知:$S^k \circ (x)=x+k \in \BF$.

\hspace*{\fill}

\textbf{证法二}:令$f_k(x)=x+k$.

1) 当$k=0$时, $f_0(x)=x=P_1^1(x)\in\IF$, 命题成立;当$k=1$时, $f_1(x)=x+1=S(x)\in\IF$, 命题成立;

2) 假设当$k=k_0(k_0\geqslant 1,k_0\in\mathbb{N})$时$f_{k_0}\in\BF$;

3) 当$k=k_0+1$时,$f_{k_0+1}=x+k_0+1=S(x+k_0)=S\circ f_{k_0}(x)=\mathrm{Comp}^1_1[S,f_{k_0}]\in\BF$.


由数学归纳法可知, 命题成立.
\end{proof}
\input{ch01/1.02.tex}
\input{ch01/1.03.tex}
\input{ch01/1.04.tex}
\input{ch01/1.05.tex}
\input{ch01/1.06.tex}
\input{ch01/1.07.tex}
\input{ch01/1.08.tex}
\input{ch01/1.09.tex}
\input{ch01/1.10.tex}
\input{ch01/1.11.tex}
\input{ch01/1.12.tex}
\input{ch01/1.13.tex}
\input{ch01/1.14.tex}
\input{ch01/1.15.tex}
\input{ch01/1.16.tex}
\input{ch01/1.17.tex}
\input{ch01/1.18.tex}
\input{ch01/1.19.tex}
\input{ch01/1.20.tex}
\input{ch01/1.21.tex}
\input{ch01/1.22.tex}
\input{ch01/1.23.tex}
\input{ch01/1.24.tex}
\input{ch01/1.25.tex}
\newpage
% Chapter 2
\section{算盘机}
\input{ch02/2.01.tex}
\input{ch02/2.02.tex}
\input{ch02/2.03.tex}
\input{ch02/2.04.tex}
\input{ch02/2.05.tex}
\newpage
% Chapter 3
\section{$\lambda$-演算}
\input{ch03/3.01.tex}
\problem{试求$SSSS$的$\bnf$.}
\begin{solution}
{\color {red} {对第一版答案做出补充说明.}}根据定义, $S \equiv \lambda xyz.xz(yz)$.注意:对于$SS$,第一个$S$和第二个$S$的变元不同.不妨令第一个$S$为$S \equiv \lambda xyz.xz(yz)$,第一个$S$为$S \equiv \lambda abf.af(bf)$.首先化简$SS$,再化简$SSSS$.

那么
    \begin{align*}
        SS&\equivbeta \lambda yz.Sz(yz)\\
        &\equivbeta \lambda yz.\ (\lambda abf.af(bf)) \ z(yz)\\
        &\equivbeta \lambda yz.\lambda f.zf((yz)f)\\
        &\equiv \lambda yzf.zf(yzf)\\
        &\equiv \lambda xyz.yz(xyz)
    \end{align*}

接下来再求$SSSS$的值,由于$SSSS\equivbeta SS(SS)$,一直进行规约化简有(注意第一个$SS$和第二个$SS$的变元不同):
    \begin{align*}
        SSSS &\equivbeta SS(SS)\\
        &\equivbeta \lambda yz.yz(SSyz)\\
        &\equivbeta \lambda yz.yz(\lambda f.zf(yzf))\\
        &\equiv \lambda xy.xy(\lambda z.yz(xyz))\tag{*}
    \end{align*}

    到了(*)式后, 已经没有含有$\beta$-可约式的子项, 故(*)式即为$SSSS$的$\bnf$.
\end{solution}
\input{ch03/3.03.tex}
\problem{设$F\in\Lambda$呈形$\lambda x.M$, 证明:\\
    (1) $\lambda z.Fz\equivbeta F$;\\
    (2) $\lambda z.yz\neq_\beta y$. \\
注意, 对于一般的$F$, $\lambda z.Fz\neq_\beta F$, 但$\lambda z.Fz=_{\eta} F$.}
\begin{proof}
{\color {red} {对第一版答案做出了补充说明.}}呈形英文: is of the form,即表示其表达式呈现出这样.
    $\lambda z.Fz\equiv \lambda z.(\lambda x.M)z\equivbeta \lambda z.M[x:=z] \equiv F \quad \text{($\lambda z.M[x:=z]$ 呈现出 $\lambda x.M$ 形状,故其恒等于$F$)}$.
	
	$\lambda z.yz$本身已是$\bnf$, 其无法$\beta$-归约到$y$.
\end{proof}
\problem{证明二元不动点定理: 对于任何$F,G\in\Lambda$, 存在$X,Y\in\Lambda$, 满足$$FXY=X,$$ $$GXY=Y.$$}
\begin{proof}
    根据一元不动点定理, 一定存在$Y$使得$GXY=Y$, 这可以表示为$Y=\bm{\Theta}(GX)$, 其中$\bm{\Theta}$为不动点组合子.
	
	那么现在只要证明, 对于任何$F,G\in\Lambda$, 存在$X\in\Lambda$, 满足$FX(\bm{\Theta}(GX))=X$.
	
	而 $FX(\bm{\Theta}(GX))=X\Rightarrow (\lambda x.Fx(\bm{\Theta}(Gx)))X=X$ (根据($\beta),(\sigma),(\tau$)), 于是$$X=\bm{\Theta}(\lambda x.Fx\bm{\Theta}(Gx)).$$
	
	再将求得的$X$解带入到$Y$中,消去$X$,有:
	\begin{align*}
		X=&\bm{\Theta}(\lambda x.Fx(\bm{\Theta}(Gx))),\\
		Y=&\bm{\Theta}(G\bm{\Theta}(\lambda x.Fx(\bm{\Theta}(Gx))))
	\end{align*}
	
	此时的$X$和$Y$对任何$F,G\in\Lambda$满足$FXY=X, GXY=Y$.
\end{proof}
\input{ch03/3.06.tex}
\input{ch03/3.07.tex}
\input{ch03/3.08.tex}
\input{ch03/3.09.tex}
\input{ch03/3.10.tex}
\input{ch03/3.11.tex}
\problem{证明: 对于任何$M,N\in\Lambda$, 若$M\equivbeta N$, 则存在$T$使$M\manystepbeta T$且$N\manystepbeta T$. 这就是对于$\equivbeta$的CR性质.}
\begin{proof}{\color {red} {对第一版答案做出了补充说明.}}

    \textbf{方法一}:由习题3.9知:$\exists P_0, \cdots , P_n$使$M \equiv P_0, N \equiv P_n$且$\forall i < n$有$P_i \rightarrow_{\beta} P_{i+1}$或$P_i \leftarrow_{\beta} P_{i+1}$.
    
    下面证明:
    $$\exists T_i \  s.t. \  P_0 \twoheadrightarrow_{\beta} T_i \ and \  P_i \twoheadrightarrow_{\beta} T_i \quad(\star)$$
    
    对$i$做归纳证明.
    
    1) 当$i=0$时,取$T_0$为$M$即可,此时$(\star)$成立;
    
    2) 假设$i=k$时,$\exists T_k \  s.t. \  P_0 \twoheadrightarrow_{\beta} T_k \ and \  P_k \twoheadrightarrow_{\beta} T_k $;
    
    3) 当$i=k+1<n$时,有$P_0 \twoheadrightarrow_{\beta} T_k \ and \  P_k \twoheadrightarrow_{\beta} T_k$(归纳假设),则
    
        情况1:$P_k \twoheadrightarrow_{\beta} P_{k+1}$,从而由CR性质,$\exists T_{k+1} \ s.t. \  T_k \twoheadrightarrow_{\beta} T_{k+1} \ and \ P_{k+1} \twoheadrightarrow_{\beta} T_{k+1}$,从而$(\star)$成立;
    
        情况2:$P_k \leftarrow_{\beta} P_{k+1}$,取$T_{k+1}$为$P_k$即可,此时$(\star)$也成立.
    
    于是归纳完成.
    
    因此有$\exists T_n \  s.t. \  P_0 \twoheadrightarrow_{\beta} T_n \ and \  P_n \twoheadrightarrow_{\beta} T_n $,取$T$为$T_n$即有$M\twoheadrightarrow_{\beta} T \ and \ N \twoheadrightarrow_{\beta} T$.
    
    %换行
    \hspace*{\fill}
    
    \textbf{方法二}:
    $M\equivbeta N$蕴含
	$$(M,N)\in\bigcup_{i\in\mathbb{N}}(\to_{\beta}\cup\leftarrow_{\beta})^k.$$
	
	当$k=0$时, $M\equivbeta N$.
	假设对所有$(M,N)\in(\to_{\beta}\cup\leftarrow_{\beta})^k$, 存在$T\in\Lambda$使得$M\manystepbeta T$且$N\manystepbeta T$.
	
	那么当$(M,N)\in(\to_{\beta}\cup\leftarrow_{\beta})^{k+1}$时, 要么有$M\onestepbeta P=_\beta N$, 要么有$M\leftarrow_\beta P=_\beta N$, 其中$(P,N)\in(\to_{\beta}\cup\leftarrow_{\beta})^k$. 那么存在$T_0$使得$P\manystepbeta T_0$且$N\manystepbeta T_0$. 由于$\manystepbeta$是传递的, 所以$M\manystepbeta T_0$.
	
	由于$\manystepbeta$的CR性质, $P\manystepbeta M$且$P\manystepbeta T_0$可得到, 存在$T\in\Lambda$使得$M\manystepbeta T$且$T_0\manystepbeta T$. 由于$\manystepbeta$是传递的, 有$N\manystepbeta T_0$且$T_0\manystepbeta T$, 因此$N\manystepbeta T$.
	
	因此对所有的$k\in\mathbb{N}$, 这样的$T$都存在. 命题成立.
\end{proof}
\problem{证明: 若在系统$\lambda\beta$中加入如下公理
$$\textrm{(A)}\qquad\lambda xy.x=\lambda xy.y,$$
则对任何的$M,N\in\Lambda$, $\lambda\beta+(A)\vdash M=N$.}

\begin{proof}{\color {red} {对第一版答案做出了补充.}}
碰到这种题目,如果有两个以上的不同变元,使用标准组合子$\mathbf{I}, \mathbf{K}, \mathbf{K^*}, \mathbf{S}$(page 78)来推导即可.

证明方法一:对于所有的$M,N\in\Lambda$,
$$ \lambda xy.x=\lambda xy.y $$
$$ \Rightarrow (\lambda xy.x)MN=(\lambda xy.y)MN $$
$$ \Rightarrow M=N $$

\textbf{证明方法二}(推荐):
$$ \lambda xy.x=\lambda xy.y $$
$$ \Rightarrow (\lambda xy.x)\mathbf{I}M=(\lambda xy.y)\mathbf{I}M $$
$$ \Rightarrow \mathbf{I}=M $$

同理,可以推导出:\qquad \qquad \qquad \qquad \quad $ \mathbf{I}=N $
$$ \Rightarrow M=N $$

故有$\lambda\beta+(A)\vdash M=N$.

\end{proof}
\input{ch03/3.14.tex}
\input{ch03/3.15.tex}
\problem{试找出$A\in\Lambdaclosed$使$A$ $\lambda$-定义函数$f(x,y)=x+y$.}
\begin{solution}
{\color {red} {对第一版答案做出了修正.}}
只要知道$\lambda fx.f^n x = \lambda fx. \churchnumber{n}fx$就好做了.因为$\churchnumber{n} \equiv \lambda ab.a^nb$,于是$\lambda fx. \churchnumber{n}fx=\lambda fx. ((\lambda ab.a^nb)fx)=\lambda fx. f^nx$.

构造$A \churchnumber{n} \churchnumber{m} = \churchnumber{n+m} =\lambda fx. f^{n+m}x= \lambda fx. f^n (f^m x)$.

\[
 \begin{array}{rcl}
  \lambda fx. f^n (f^m x) & = & \lambda fx. \churchnumber{n}f (\churchnumber{m} f x) \\
  & = & (\lambda vfx. \churchnumber{n}f (vf x))\churchnumber{m} \\
  & = & (\lambda uvfx. uf (vf x))\churchnumber{n}\churchnumber{m}.
 \end{array}
\]

因此, $A \equiv \lambda uvfx. uf(vfx)$ $\lambda$-定义函数$f(x,y)=x+y$.
\end{solution}
\input{ch03/3.17.tex}
\input{ch03/3.18.tex}
\input{ch03/3.19.tex}
\problem{构造$F\in\Lambdaclosed$使得对于任何$n\in\mathbb{N}$,
$$F\churchnumber{n}\equivbeta\churchnumber{2^n}.$$}
\begin{solution}

{\color {red} {对第一版答案进行了补充说明.}}

方法一:由引理3.33(page 97)得,存在$D\in\Lambdaclosed$,使得其能够完成if-else功能.

注意到$\mathrm{Exp} \equiv \lambda xy.yx $能够实现一个指数函数功能,但是指数的取值为$\mathbb{N}^{*}$.这一题给定的指数范围是$\mathbb{N}$,于是使用$D$来实现if-else判断.构造结果如下:
    $$F\equiv \lambda x.Dx\churchnumber{1}(\mathrm{Exp}\churchnumber{2}x).$$
    
方法二:设$H$ $\lambda-$定义函数$h(x)=2x$.
\[
 \begin{array}{rcl}
  \lambda fx. f^n (f^n x) & = & \lambda fx. \churchnumber{n}f (\churchnumber{n} f x) \\
  & = & (\lambda vfx. vf (vf x))\churchnumber{m} .
 \end{array}
\]

于是$H \equiv \lambda vfx.vf(vfx) \  \lambda-$定义函数$h(x)=2x$.注意到:
$$\churchnumber{2^x} = \churchnumber{2 \cdot 2^{x-1}} = H\churchnumber{2^{x-1}}=H(H\churchnumber{2^{x-2}})=H^x\churchnumber{2^{0}}=H^x\churchnumber{1}=\churchnumber{x}F\churchnumber{1}=(\lambda x.xH\churchnumber{1})\churchnumber{x}$$

令$F\equiv \lambda x.xH\churchnumber{1}$,则对于任何$n\in\mathbb{N},F\churchnumber{n}\equivbeta\churchnumber{2^n}.$

\end{solution}
\problem{设$f,g:\mathbb{N}\to\mathbb{N}, f=\mathrm{Itw}[g]$, 即
$$\protect\begin{aligned}
   f(0)&=0,\\
   f(n+1)&=g(f(n)),
\protect\end{aligned}$$
且$G\in\Lambdaclosed$ $\lambda$-定义函数$g$. 试求$F\in\Lambdaclosed$使得$F$ $\lambda$-定义函数$f$.}

\begin{proof}


\[
  \begin{array}{rcl}
  F\churchnumber{n} & = & D\churchnumber{n}\churchnumber{0} G(F(\mathrm{pred} \churchnumber{n})) \\
  & = & \lambda x. Dx\churchnumber{0} G(F(\mathrm{pred}\ x))  \churchnumber{n} \\
  & = & \Big(\lambda fx.\mathrm{D}x\churchnumber{0}G \big(f(\mathrm{pred}\ x) \big)\Big) F \churchnumber{n}. \\
  \end{array}
\]



由不动点定理可以求出$F$.于是$F\equiv \mathbf{Y} \bigg( \Big(\lambda fx.\mathrm{D}x\churchnumber{0}G \big(f(\mathrm{pred}\ x) \big)\Big) \bigg)$ $\lambda$-定义函数$f$.
\end{proof}
\problem{证明引理3.39.}

引理3.39表述为:

存在一般递归函数$\mathrm{var,app,abs,num}:\mathbb{N}\to\mathbb{N}$使得:
\begin{enumerate}[label=(\arabic*)]
    \item $\forall n\in\mathbb{N}. \mathrm{var}(n)=\sharp (v^{(n)});$\label{def:var}
    \item $\forall M, N\in \Lambda. \mathrm{app}(\sharp M,\sharp N)=\sharp (MN);$\label{def:app}
    \item $\forall x\in\nabla, M\in\Lambda. \mathrm{abs}(\sharp x, \sharp M) = \sharp(\lambda x. M);$]\label{def:abs}
    \item $\forall n\in\mathbb{N}. \text{num}(n)=\sharp\churchnumber{n}.$
\end{enumerate}

\begin{proof}{\color {red} {对第一版答案进行了补充说明.}}取$[x,y]=2^x \cdot 3^y, \Pi_1=ep_1, \Pi_2=ep_1$,从而$[\cdot,\cdot],\Pi_1,\Pi_2 \in \EF$.

\begin{enumerate}[label=(\arabic*)]
    \item 由于$v^{(n)}=[0,n] \in \EF$, 取var($n$)=$[0,n]$即可;
    \item 取app($m,n$)=$[1,[m,n]] \in \EF$, 则$\text{app}(\sharp M,\sharp N)=\sharp (MN)$;
    \item 取abs($n,m$)=$[2,[\text{var}(n),m]] \in \EF$, 易验证$\text{abs}(\sharp v^{(n)}, \sharp M) = \sharp(\lambda v^{(n)}. M)$;
    \item 下面证$\forall n\in\mathbb{N}. \text{num}(n)=\sharp\churchnumber{n}$,且num$(n)$是递归的:
        $$\begin{aligned}
          \churchnumber{n+1}&=\sharp(\lambda fx. f^{n+1}x)\\
          &=\Big[2, \Big[\sharp f, [2, [\sharp x, \sharp f^{n+1}x]]\Big]\Big]\\
          &=\Big[2, \Big[\sharp f, \big[2, \big[\sharp x, [1,[\sharp f, \sharp f^n x]]\big]\big]\Big]\Big]
        \end{aligned}$$
        
        $$\begin{aligned}
          \churchnumber{n}&=\sharp(\lambda fx. f^{n}x)\\
          &=\Big[2, \Big[\sharp f, [2, [\sharp x, \sharp f^{n}x]]\Big]\Big]\\
        \end{aligned}$$
        故有:
        $$\sharp f^{n}x = \Pi_2(\Pi_2(\Pi_2(\Pi_2(\sharp \churchnumber{n}))))=\Pi_2^4(\sharp \churchnumber{n})$$
        于是
        $$\churchnumber{n+1} = \Big[2, \Big[\sharp f, \big[2, \big[\sharp x, [1,[\sharp f, \Pi_2^4(\sharp \churchnumber{n})]]\big]\big]\Big]\Big]$$
        令$h(z) = \Big[2, \Big[\sharp f, \big[2, \big[\sharp x, [1,[\sharp f, \Pi_2^4(z)]]\big]\big]\Big]\Big]$,$\text{num}(0)=\churchnumber{0}$,于是:
        $$\text{num}(n+1)=h(\text{num}(n))$$
        因此num $\in \PRF$,$\text{num}(n)=\churchnumber{n}$.
    
\end{enumerate}

\end{proof}
\problem{设$f(n)$为习题1.16中定义的函数,试构造$F\in\Lambdaclosed$使 $F\churchnumber{n}=\churchnumber{f(n)}$对$n \in \mathbb{N}^{+}$成立.}

\begin{proof}
{\color {red} {对第一版答案做出了修正.}}
习题1.16定义的函数$f(n)$为:
$$\protect\begin{aligned}
    f(0)&=0,\\
    f(n)&=\protect\underbrace{n^{\cdot^{\cdot^{\cdot n}}}}_{n\textrm{个}n},
\protect\end{aligned}$$

令$w_n=\lambda x.\underbrace{x \cdots x}_{n\textrm{个}x}$,$n \in \mathbb{N}^{+}$,$x$为$v^{(0)}$.令$[x,y]=2^x \cdot 3^y$, $\Pi_2 [x,y] = y$.于是
$$\sharp v^{(0)}=[0,0]=1$$

则$\sharp x=[0,0]=1$.

令$h(n)= \sharp w_n (n \ge 1)$,于是
$$h(n)=\sharp w_n=[2,[\sharp x, \sharp \underbrace{x \cdots x}_{n\textrm{个}x}]]$$
补充定义$h(0)=0$.现证明$h(n) \in \PRF $:
\[
 \begin{array}{rcl}
  h(n+1) & = & \sharp w_{n+1} \\
  & = & [2,[\sharp x, \sharp \underbrace{x \cdots x}_{n+1\textrm{个}x}]] \\
  & = & [2,[\sharp x, [1, [\sharp \underbrace{x \cdots x}_{n\textrm{个}x},  \sharp x]]]] \ \ \ (\textrm{according\ to\ definition\ 3.36(2)})
 \end{array}
\]
注意到:$$\sharp \underbrace{x \cdots x}_{n\textrm{个}x}= \Pi _2 (\Pi _2 (h(n)))=\Pi _2^2 (h(n))$$

根据$\sharp x=[0,0]=1$和$h(n+1)=[2,[\sharp x, [1, [\sharp \underbrace{x \cdots x}_{n\textrm{个}x},  \sharp x]]]]$,有:

$$h(n+1)=[2,[1,[1,[\Pi _2^2(h(n)),1]]]]$$

于是$h(n) \in \PRF $.

由定理3.41(page 101),存在枚举子$E$,使得:
$$E(H\churchnumber{n})=E\churchnumber{w_n}=w_n $$

取$M \equiv \lambda z.(E(Hz))z $,于是:

\[
 \begin{array}{rcl}
  M\churchnumber{n} & = & (E(H\churchnumber{n}))\churchnumber{n} \\
  & = & w_n\churchnumber{n} \\
  & = & \underbrace{\churchnumber{n} \cdots \churchnumber{n}}_{n\textrm{个}n}\\
  & = & \churchnumber{\underbrace{n^{\cdot^{\cdot^{\cdot n}}}}_{n\textrm{个}n}} \ \ (n \ge 1)
 \end{array}
\]

由于$n \in \mathbb{N}$,因此$M$此时不能完全$\lambda -\textrm{可定义}\  f(n)$,缺少$n=0$的情况.由引理3.33,令$D \equiv [U_3^3, U_1^2]$.

取$L \equiv \lambda z.Dz\churchnumber{0}(Mz)$,此时$L \  \lambda-\textrm{可定义}\  f(n)$.

\end{proof}
\problem{构造$H\in\Lambdaclosed$, 使得对于任意$n\in\mathbb{N}, x_1,\cdots, x_n\in\Lambda$, 有$$H\churchnumber{n}x_1\cdots x_n\equivbeta \lambda z.zx_1\cdots x_n.$$}
\begin{proof}
{\color {red} {对第一版答案做出了补充.}}
    
    证法一(严格证法):令$L_n\equiv[x_1,\cdots,x_n]\equiv\lambda z.z x_1\cdots x_n \ (n \in \mathbb{N})$.这里$x_i$为第$i$个变元,$z$为$v^{(0)}$,则$\sharp z = 1$.
    
    设$l(n)=\sharp L_n$,约定$l(0)=0$.下面证$l(n) \in \GRF$:
    
    令$h(n)=\sharp z x_1 \cdots x_n$,则$l(n)=[2,[\sharp z, \sharp z x_1 \cdots x_n]]=[2,[1,h(n)]]$.
    $$h(1)=\sharp z x_1 = [1,[1,\sharp z_1]] = [1,[1,[0,1]]]$$
    $$h(n+1)=\sharp z x_1 \cdots x_n x_{n+1}=[2,[h(n),\sharp x_{n+1}]]=[2,[h(n),[0,n+1]]]$$
    
    补充$h(0)=0$,故$h(n)\in \PRF$.
    
    于是$$l(n)=\protect\begin{cases}
    0, & if\ n=0,\\
    [2,[1,h(n)]], & otherwise.
\protect\end{cases}$$

    因此$l(n) \in \PRF$.
    
    令$M_n= \lambda x_1 \cdots x_n.[x_1, \cdots, x_n]$,$g(n)=\sharp M_n$,于是有:
    $$g(n)=\sharp M_n=[2,[\sharp x_1,[2,\cdots[2,[x_n,l(n)]]\cdots]]]$$
    
    其中$\sharp x_i = [0,i]$.令$f(i,y)=[2,[[0,i],y]] \in \PRF$.则$$g(n)=f(1,f(2,\cdots,f(n-1,f(n,l(n)))\cdots))$$
    
    类比于习题1.17,可证$g(n)\in \PRF$.故存在$G \in \Lambdaclosed, G\ \lambda-$定义$g$.从而$G\churchnumber{n}=\churchnumber{M_n}$.
    
    由定理3.21(page 101),存在$E$,使得$E(G\churchnumber{n})=E(\churchnumber{M_n})=M_n$,令$H\equiv\lambda z.E(Gz)$,从而:
    
    $H\churchnumber{n}x_1\cdots x_n = \lambda z.E(Gz)x_1\cdots x_n=M_n x_1 \cdots x_n=[x_1, \cdots, x_n]=\lambda z.z x_1 \cdots x_n$.
    
    \hspace*{\fill}
    
    证法二(简单证法):
    
    可知我们需要找到$H$使得$H\churchnumber{n}=\lambda x_1x_2\cdots x_nz. zx_1\cdots x_n$.
    
    我们可以令$M_n=\lambda x_1x_2\cdots x_nz. zx_1\cdots x_n$, 那么其编码
    $$g(n)=\sharp M_n=[2,[\sharp x_1, \sharp (\lambda x_2\cdots x_n.zx_1\cdots x_n)]]$$
    
    而$\sharp (zx_1\cdots x_n)$是递归的, 因此$g$是递归的. 设$G\in\Lambdaclosed$ $\lambda$-定义了$g$, 从而$G\churchnumber{n}\equivbeta\churchnumber{M_n}$, 因此
    $$E(G\churchnumber{n})\equivbeta E\churchnumber{M_n}\equivbeta M_n$$
    
    取$H\equiv \lambda x.E(Gx)$即可.
    
    
    
\end{proof}
\problem{证明: 存在$\Theta_2 \in \Lambdaclosed$, 使得对于任意$F\in \Lambdaclosed$, 有$$\Theta_2 \churchnumber{F}\equivbeta F\churchnumber{\Theta_2 \churchnumber{F}}.$$}
\begin{proof}
{\color {red} {对第一版答案进行了修正.}}    
    令$W \equiv \lambda xy.Ey(\text{App} (\text{App} \ x(\text{Num} \ x))(\text{Num} \ y))$, $\Theta_2=W \churchnumber{W}$, 对于$F\in \Lambdaclosed$有$E\churchnumber{F}=F$,这里$E$是枚举子.则:
    \begin{align*}
        \Theta_2 \churchnumber{F} &\equiv W \churchnumber{W}\churchnumber{F}\\
        &\equivbeta E\churchnumber{F}(\text{App} (\text{App} \ \churchnumber{W}(\text{Num} \ \churchnumber{W}))(\text{Num} \ \churchnumber{F}))\\
        &\equivbeta F(\text{App}(\text{App} \churchnumber{W} \churchnumber{\churchnumber{W}}))\churchnumber{\churchnumber{F}} \\
        &\equivbeta F \text{App} \churchnumber{W \churchnumber{W}} \churchnumber{\churchnumber{F}}\\
        &\equiv F\churchnumber{\Theta_2\churchnumber{F}}
    \end{align*}

    因此结论成立.
\end{proof}
\input{ch03/3.26.tex}
\newpage
\section{组合逻辑}
\input{ch04/4.01.tex}
\input{ch04/4.02.tex}
\input{ch04/4.03.tex}
\input{ch04/4.04.tex}
\input{ch04/4.05.tex}
\input{ch04/4.06.tex}
\input{ch04/4.07.tex}
\input{ch04/4.08.tex}
\input{ch04/4.09.tex}
\input{ch04/4.10.tex}
\newpage
\section{Turing机}
\input{ch05/5.01.tex}
\problem{构造机器$\machine{copy_1}$使$\machine{copy_1}|0\underset{\uparrow}{1^x}0\cdots \twoheadrightarrow 01^x0\underset{\uparrow}{1^x}0\cdots$.}
\begin{solution}

{\color {red} {对第一版答案做出了修正.}} 第一版答案求解思路是每删除一个1,在其后面对应的位置加两次1,这样做确实能够实现拷贝功能,但是其最终的输出开头的0数量超过1个,与题意不符.

求解思路:$0\underset{\uparrow}1^x0 \cdots \twoheadrightarrow 01\underset{\uparrow}0^{x-1}010 \cdots \twoheadrightarrow 0\underset{\uparrow}11^{x-1}011^{x-1}0\cdots$.注意考虑特殊情况,即当$x=0$时候,输出为$0\underset{\uparrow}101 \cdots$.

表\ref{tab:sol5.2}定义了机器$\machine{copy_1}$:
\begin{table}[!htbp]
\centering
\caption{机器$\machine{copy_1}$}
\label{tab:sol5.2}
\begin{tabularx}{\textwidth}{Y|Y|Y}
\thickhline
    &  0    &      1   \\
\hline
1   &  &   $1R2$   \\
\hline
2   & $0R5$ &   $0R3$   \\ %特殊情况考虑,即010...
\hline
3   & $0R4$ &   $0R3$   \\ %删完所有的1
\hline
4   & $1L7$ &   $1R5$   \\ %如果刚开始赋值则补1
\hline
5   & $1L6$ &   $1R5$   \\ %如果之前复制过则在后面补1
\hline
6   & $0L7$ &   $1L6$   \\ %回到复制1的前面
\hline
7   & $0L8$ &   $1R11$   \\ %需要判断是不是复制完了
\hline
8   & $0L8$ &   $1R9$   \\ %没有复制完
\hline
9   & $1R10$ &   $ $   \\ %开始恢复之前删的1
\hline
10   & $0R10$ &   $1R5$   \\ %删去的1恢复1个则需要在复制的区域加1
\hline
11   & $0R12$ &   $ $   \\ %复制完成的话则将指针回到标准输出位置
\thickhline
\end{tabularx}
\end{table}

输入$0\underset{\uparrow}1^x0 \cdots$,输出$0\underset{\uparrow}1^x0 1^x0\cdots$.
\end{solution}
\input{ch05/5.03.tex}
\problem{构造机器计算函数$f(x)=2^x$.}
\begin{solution}
由定理5.13的证明过程, 可如此构造:令$y$恒为1,令$f(x+1,y)=g(f(x,y))$,其中$g(x)=2x$,于是有$f(x)=2^x$.

首先, 构造出初始值($y$)1, 使用机器$M_1$, 定义如表\ref{tab:sol5.4m1}:
\begin{table}[!htbp]
\centering
\caption{题5.4机器$M_1$}
\label{tab:sol5.4m1}
\begin{tabularx}{\textwidth}{Y|Y|Y}
\thickhline
    &  0    &      1   \\
\hline
1   & $0R2$ &   $1R1$   \\
\hline
2   & $1R3$ &           \\
\hline
3   & $1L4$ &           \\
\hline
4   &       &   $1L5$   \\
\hline
5   & $0L6$ &           \\
\hline
6   & $0R7$ &   $1L6$   \\
\thickhline
\end{tabularx}
\end{table}

易知$M_1|1:0\underset{\uparrow}{1}^{x+1}00\cdots\twoheadrightarrow7:0\underset{\uparrow}{1}^{x+1}01100\cdots$.

定义机器$M_2$为表\ref{tab:sol5.4m2}:
\begin{table}[!htbp]
\centering
\caption{题5.4机器$M_2$}
\label{tab:sol5.4m2}
\begin{tabularx}{\textwidth}{Y|Y|Y}
\thickhline
    &  0    &      1   \\
\hline
1   &       &   $0R2$   \\
\hline
2   & $0Ru$ &   $1R3$   \\
\hline
3   & $0R4$ &   $1R3$   \\
\thickhline
\end{tabularx}
\end{table}

易知$x>0$时$M_2|1:0\underset{\uparrow}{1}^{x+1}01100\cdots\twoheadrightarrow 4:001^{x}0\underset{\uparrow}{1}100\cdots$ (在$x=0$时输出为$u:000\underset{\uparrow}{1}100\cdots$).

令$M_3=M_2\concat \machine{double} + 3\concat \machine{compress}\concat \machine{shiftl}$,从而

若$x=0$,则$M_3|1:0\underset{\uparrow}1^{x+1}01^{y+1}0 \cdots \twoheadrightarrow v:000\underset{\uparrow}1^{y+1}0 \cdots$

若$x > 0$,则$M_3|1:0\underset{\uparrow}1^{x+1}01^{y+1}0 \cdots \twoheadrightarrow w:00\underset{\uparrow}1^{x}01^{2y+1}0 \cdots$

$M_4=\mathrm{repeat}\ M_3$,即$M_4=M_3[w:=1]$, 机器$\machine{f}=M_1\concat M_4$为所求.

\end{solution}
\problem{设机器$M_1$定义如表5.24.\\
对于输入$\bar{x}$, 求输出.}
\begin{table}[H]
\centering
\caption*{\textbf{表5.24}}
\begin{tabularx}{\textwidth}{Y|Y|Y}
\thickhline
    &  0    &      1   \\
\hline
1   & $0L3$ &   $1R2$   \\
\hline
2   & $0L3$ &   $0R1$   \\
\hline
3   & $0L3$ &   $1L3$   \\
\thickhline
\end{tabularx}
\end{table}

\begin{solution}
{\color {red} {对第一版答案做出了修正.}}
$M_1$的操作为把第偶数次读到的1写成0,读到0后一直向左倒带到第一个位置的左端(越界),从而无法读取从而停机. 因此当其输入为$\bar{x}$时, 输出为$\underset{\uparrow} \ \underbrace{0101\cdots01}_{\left\lfloor \frac{x}{2}\right\rfloor+1\text{个}01}0 \cdots$.
\end{solution}
\input{ch05/5.06.tex}
\input{ch05/5.07.tex}
\input{ch05/5.08.tex}
\input{ch05/5.09.tex}
\input{ch05/5.10.tex}
\input{ch05/5.11.tex}
\input{ch05/5.12.tex}
\problem{证明: $S=\{a_1,a_2,\cdots,a_k\}$是Turing-可计算的.}

\begin{proof}

{\color {red} {对第一版答案做出了修正.}}
构造${\chi}_S = N^2 \big( \prod ^k_{i=1}(x \dotdiv a_i) \big)$,于是有:

$${\chi}_S=
\protect\begin{cases}
    0, & \textrm{若}x \in S,\\
    1, & \textrm{否则}.
\protect\end{cases}$$

故${\chi}_S \in \EF$.由定理5.18(page \  138)知${\chi}_S$是Turing-可计算的.

\end{proof}
\input{ch05/5.14.tex}
\input{ch05/5.15.tex}
\input{ch05/5.16.tex}
\input{ch05/5.17.tex}
\input{ch05/5.18.tex}
\input{ch05/5.19.tex}
\input{ch05/5.20.tex}
\newpage
\end{document}